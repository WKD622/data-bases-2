\newpage

\section{Polecnie 9}

\subsection{TRIGER OBSŁUGUJĄCY DODANIE REZERWACJI}
Zmiana strategii obsługi redundantnego pola liczba\_wolnych\_miejsc. realizacja przy pomocy
trigerów
\subsubsection{DODAJ REZERWACJE 3}
\begin{verbatim}
CREATE PROCEDURE DODAJ_REZERWACJE_3(ID_W NUMBER, ID_O NUMBER)
AS
  BEGIN
    DECLARE
      WYCIECZKI NUMBER;
      OSOBA     NUMBER;
    BEGIN

      SELECT COUNT(W.ID_WYCIECZKI) INTO WYCIECZKI FROM WYCIECZKI W WHERE W.ID_WYCIECZKI = ID_W;
      SELECT O.ID_OSOBY INTO OSOBA FROM OSOBY O WHERE O.ID_OSOBY = ID_O;

      IF WYCIECZKI > 0 AND OSOBA > 0
      THEN
        INSERT INTO REZERWACJE (ID_WYCIECZKI, ID_OSOBY, STATUS) VALUES (ID_W, ID_O, 'N');
      END IF;
    END;
  END;
\end{verbatim}

\subsubsection{TRIGER DODAJ REZERWACJE 3}
\begin{verbatim}
CREATE OR REPLACE TRIGGER DODANIE_REZERWACJI_3
  AFTER UPDATE
  ON REZERWACJE
  FOR EACH ROW
  DECLARE
    AKTUALNA_DATA DATE;
  BEGIN
    SELECT CURRENT_DATE INTO AKTUALNA_DATA FROM DUAL;
    INSERT INTO DZIENNIK_REZERWACJI (NR_REZERWACJI, DATA, NOWY_STATUS)
    VALUES (:OLD.NR_REZERWACJI, AKTUALNA_DATA, :NEW.STATUS);
    BEGIN
      PRZELICZ;
    END;
  END;
\end{verbatim}

Ten trigger różni się od dodaj\_rezerwacje\_2 tylko dodaniem procedury przelicz, której poprawne 
działanie dowiodłem już wcześniej, z tego powodu nie będę zamieszczał zrzutu ekranu. 

\subsection{TRIGER ZMIANA STATUSU REZERWACJI}
triger obsługujący zmianę statusu
Ten sam co poprzednio, ponieważ pole liczba miejsc nie ma związku ze statusem rezerwacji.

\begin{verbatim}
CREATE OR REPLACE TRIGGER ZMIANA_STATUSU_REZERWACJI
  AFTER UPDATE
  ON REZERWACJE
  FOR EACH ROW
  DECLARE
    AKTUALNA_DATA DATE;
  BEGIN
    SELECT CURRENT_DATE INTO AKTUALNA_DATA FROM DUAL;
    INSERT INTO DZIENNIK_REZERWACJI (NR_REZERWACJI, DATA, NOWY_STATUS)
    VALUES (:OLD.NR_REZERWACJI, AKTUALNA_DATA, :NEW.STATUS);
  END;
\end{verbatim}

\subsection{ZMIEN LICZBE MIEJSC 3}
triger obsługujący zmianę liczby miejsc na poziomie wycieczki
\begin{verbatim}
CREATE PROCEDURE ZMIEN_LICZBE_MIEJSC_3(ID_WYCIECZKI_ NUMBER, NOWA_LICZBA_MIEJSC NUMBER)
AS
  BEGIN
    DECLARE
      CALKOWITA_LICZBA_MIEJSC NUMBER;
      DOSTEPNE_MIEJSCA_ NUMBER;
    BEGIN
      SELECT W.LICZBA_MIEJSC INTO CALKOWITA_LICZBA_MIEJSC FROM WYCIECZKI W WHERE W.ID_WYCIECZKI = ID_WYCIECZKI_;
      SELECT DOSTEPNE_MIEJSCA(ID_WYCIECZKI_) INTO DOSTEPNE_MIEJSCA_ FROM DUAL;
      IF (NOWA_LICZBA_MIEJSC >= (CALKOWITA_LICZBA_MIEJSC - DOSTEPNE_MIEJSCA_))
      THEN
        UPDATE WYCIECZKI W SET W.LICZBA_MIEJSC = NOWA_LICZBA_MIEJSC WHERE W.ID_WYCIECZKI = ID_WYCIECZKI;
      END IF;
    END;
  END;
\end{verbatim}

\subsection{TRIGER ZMIANA LICZBY MIEJSC}
\begin{verbatim}
CREATE OR REPLACE TRIGGER ZMIANA_LICZBY_MIEJSC
  before UPDATE
  ON WYCIECZKI
  FOR EACH ROW
  BEGIN
    declare
      VAL number;
    begin
      VAL := :old.LICZBA_MIEJSC - :old.LICZBA_WOLNYCH_MIEJSC;
      :new.LICZBA_WOLNYCH_MIEJSC := :new.LICZBA_MIEJSC - VAL;
    END;
  end;
\end{verbatim} 



Oczywiście po wprowadzeniu tej zmiany należy uaktualnić procedury modyfikujące dane.
Najlepiej to zrobić tworząc nowe wersje (np. z sufiksem 3)
